\thispagestyle{empty}

\section*{Anotácia}

\begin{minipage}[t]{1\columnwidth}%
Slovenská technická univerzita v Bratislave

FAKULTA INFORMATIKY A INFORMAČNÝCH TECHNOLÓGIÍ

Študijný program: \hspace{1.4cm} \myStudyProgramSK\\

\noindent
Autor: \hspace{3.5cm} \myName \\
Bakalárska práca: \hspace{1.5cm} \myTitleSK \\
Vedúci bakalárskej práce: \hspace{0.12cm} \mySupervisor \\
\noindent
\myDateSK%
\end{minipage}

\bigskip{}

% Vlastný text anotácie, 150-200 slov.
Táto práca analyzuje riziká spojené s ``prompt engineeringom'' najmä tie, ktoré sa týkajú etiky a bezpečnosti. Cielom tejto práce je vytvoriť usmernenie predovšetkým pre nových resp. neskúsených používateľov v tejto oblasti, ako by sa mali správať v súlade s etikou a bezpečnosťou. Toto usmernenie vyplýva najmä z nariadenia Európskeho parlamentu a rady (EÚ) bežne označované ako ``Akt o umelej inteligencii'', ktorý stanovuje pravidlá etického správania. V práci sú taktiež spomenuté metódy tzv. jailbreakingu systémov umelej inteligencie, čiže obídenia bezpečnostných opatrení stanovených vývojárom daného modelu. Taktiež sú v práci spomenuté vykonané experimenty s jailbreakingom a vyhodnotené ich výsledky. V teoretickej časti, sa okrem iného zaoberáme aj vysvetlením bežných pojmov spojených s umelou inteligenciou a legislatívou tejto problematiky vo viacerých krajinách než len v krajinách Európskej Únie. V neposlednom rade budú spomenuté metódy filtrovania potenciálne nebezpečného obsahu, ktorý by mohol byť generovaný pomocou modelov umelej inteligencie ako aj iné ochranné mechanizmy.


\newpage{}\thispagestyle{empty}

\newpage
\thispagestyle{empty}
\mbox{}
\newpage

\thispagestyle{empty}

\section*{Annotation}

\begin{minipage}[t]{1\columnwidth}%
Slovak University of Technology Bratislava

FACULTY OF INFORMATICS AND INFORMATION TECHNOLOGIES

Degree course: \hspace{1.4cm} \myStudyProgramEN\\

\noindent
Author: \hspace{2.6cm} \myName \\
Bachelor's Thesis: \hspace{0.75cm} \myTitleEN \\
Supervisor: \hspace{2cm} \mySupervisor \\
\noindent
\myDateEN%
\end{minipage}

\bigskip{}

% Annotation text in English, 150-200 words.
This thesis analyzes the risks associated with ``prompt engineering'', especially those related to ethics and security. The aim of this work is to provide guidance, especially for new or inexperienced users in this field, on how to behave in accordance with ethics and security. This guidance stems in particular from the Regulation of the European Parliament and of the Council (of EU) commonly referred to as the ``Artificial Intelligence Act'', which lays down rules for ethical behavior. The thesis also mentions methods of so-called jailbreaking of AI systems, i.e. bypassing the security measures set by the developer of a given model. In the thesis are also mentioned the experiments carried out with jailbreaking, and their results are evaluated. In the theoretical part, among other things, we deal with the explanation of common terms associated with artificial intelligence and the legislation of this issue in more countries than just the European Union countries. Last but not least, methods of filtering potentially dangerous content that could be generated by AI models as well as other protection mechanisms will be mentioned.

\newpage{}\thispagestyle{empty}\medskip{}


\newpage{}

\newpage
\thispagestyle{empty}
\mbox{}
\newpage
