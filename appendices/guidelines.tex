\setcounter{figure}{0}
\setcounter{table}{0}
\setcounter{section}{0}
\setcounter{listing}{0}

\chapter{Guidelines for users \label{cha:guidelines}}
\renewcommand{\thepage}{B-\arabic{page}}

% priloha guidelines (8 normostran)

% custom ref segment for the guidelines in appendix
\begin{refsegment}
% custom mini Table of contents for the guidelines in appendix
\minitoc
\clearpage

\section{Overview}
These guidelines were created to establish ethical handling of artificial intelligence, primarily for non-expert users of the general public and developers. The guidelines explain the basics of prompt engineering for novice users, and later focus on recommendations for the ethical and fair use of AI systems. In the end, concrete examples for identifying AI-generated content and beneficial use of AI are presented.

These guidelines are grounded in the EU Ethics Guidelines for Trustworthy AI~\cite{AIHLEG2019}, which were created by the independent High-Level Expert Group on AI appointed by the European Commission.

\section{Introduction to prompt engineering}
This section focuses mostly on non-experts or novice users in this field. As explained in the thesis, prompt engineering involves designing and optimizing text instructions called prompts, which are mainly used to communicate with chatbots that use LLMs in their background. When crafting these prompts, we want to be precise in writing them. To generate a desired output the prompts need to have certain quality and clarity, and they need to be specific as the LLMs are prone to generating vague, irrelevant, incorrect and unnecessarily long responses that often do not include the desired answers.

Many people think that prompt engineering is closely related to computer science and programming. However, a lack of expertise in these areas should not discourage people from using LLMs, as such knowledge is not essential.

Now, let us go through some prompt techniques that could help achieve better results from LLM prompting:
Test~\cite{PromptingTechniques}.

\textbf{test}

test

\textbf{test}

test

\textbf{test}

test

\textbf{test}

% Understanding basic prompt engineering techniques does not require deep programming knowledge. It involves learning how to be precise with your language, provide sufficient context, specify the desired format or tone of the output, and sometimes, instruct the model on what *not* to do. For example, instead of asking, "Tell me about dogs," a more effective prompt might be, "Write a 500-word article for a children's magazine about the three most popular dog breeds in the UK, focusing on their temperament and exercise needs." This section will explore various prompting strategies, such as providing examples (few-shot prompting), instructing the AI to adopt a specific persona (e.g., "Act as a historian"), or breaking down complex tasks into smaller, manageable prompted steps. We will also touch upon how prompts can inadvertently introduce bias or elicit undesirable content if not carefully considered, laying the groundwork for the ethical considerations discussed later. The goal is to empower every user to communicate more effectively with AI, making these powerful tools more accessible and useful for everyone, while also fostering an initial awareness of the responsibilities that come with directing them.



% \section{Interacting with AI generated content}
%     \subsection*{Identifying AI-generated content online}

%     \subsection*{Beneficial use of AI}
%         % helpful AI asistant (persona ?)
%         \subsubsection*{AI for creativity}
%         \subsubsection*{AI for explanation and learning}



% \section{Recommendations for the ethical and fair use of LLMs}
%     % for general public and developers

%     \subsection*{Human agency and oversight}
%     \subsection*{Technical robustness and safety}
%     \subsection*{Privacy and data governance}
%     % \subsection*{Transparency} toto riesit v dalsej sekcii
%     \subsection*{Diversity, non-discrimination and fairness}
%     \subsection*{Societal and environmental well-being}
%     \subsection*{Accountability}

% \section{Recommendations for improvements to transparency of AI systems}
%     % for developers
%     \subsection*{Transparency} % (note: explain using reasoning models (use their output to fact-check themselves))

% \section{Conclusion}
%     \subsection*{Summary}

% Bibliography
\clearpage
\printbibliography[heading=subbibliography,segment=\therefsegment,resetnumbers=true]

\end{refsegment}