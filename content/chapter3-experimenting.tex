\chapter{Experimenting}

% V tejto kapitole sa budem pokúšať experimentovať s novými a aj zároveň starými LLMs ktoré sa pokúsim brejknúť a ukážem na nich ich etickosť a zároveň ehm či majú nejaké iné obmedzenia napríklad deepseek tajomán square že čo sa tam stalo že to blokuje a takéto veci, chatgpt dan prompts, gemini a ostatné

In this chapter, we will cover experiments that were performed to analyze the ethical and security aspects of various LLMs. The focus will be on evaluating their resilience against jailbreaks and identifying potential biases and censorship patterns.

The selected models for these experiments include:
\begin{itemize}
    \item OpenAI ChatGPT
    \item Microsoft Copilot
    \item DeepSeek V3
    \item Perplexity
    % \item Google Gemini % Unable to create a testing account without a phone number
    % \item Anthropic Claude Sonnet % Account creation not possible at the moment
    % \item Meta Llama % Requires a Facebook/Instagram account, which adds unnecessary complexity
\end{itemize}

These models were chosen specifically because of the diverse nature of their providers and the differences between the models themselves. One exception is ChatGPT and Microsoft Copilot. They are based basically on the same technology because Microsoft Copilot is using ChatGPT under the hood. We have chosen two of the same models by different companies to examine the differences between their respective content moderation.



\section{Jailbreaking}
TBD
% experimenting with jailbreaks

% how they did it in the past, how they do it now

% povedat ktore modely su najviac eticke

% Pri experimentovani okrem jailbreakingu urobiť aj ukazku cenzury napr. China deepseek what happened at tiaman square