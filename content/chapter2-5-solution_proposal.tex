\chapter{Solution Proposal}
% navrh riesenia (1 - 1,5 normostrany)
% pozriet si prace na crzp ako to ma vyzerat
%   - povedat ze: definovanie problematiky, nacrh riesenia, co treba iresit, a ja konkretne co idem riesie
%   - co je vsetko zle v tomto svete, malo by sa to riesit, a potom ja co konkretne idem riesit

The goal of this thesis is to establish guidelines for ethical handling of artificial intelligence, primarily for novice users and the general public. In our analysis in Sections~\ref{sec:risks}, \ref{sec:content_moderation}, \ref{sec:methods_of_attacks}, we identified a great number of risks of using AI solutions and potential ways to misuse them for adversary purposes. These risks raise questions about the credibility and fair use of large language models. Lack of transparency of these systems, mainly due to the current state of global legislation and because most systems are what is called ''black-box'' which Collins Dictionary~\cite{Collins_BlackBox} defines as "anything having a complex function that can be observed but whose inner workings are mysterious or unknown", contribute to the need of these guidelines.

The guidelines will consist of three main sections:
\begin{enumerate}
    \item introduction to prompt engineering
    \item recommendations for the ethical and fair use of large language models and security measures
    \item improvements to transparency of AI systems % (note: explain using reasoning models (use their output to fact-check themselves))
\end{enumerate}

In the first section of the guidelines, our aim is that nonexperts in the AI field will be able to understand the topic of AI and mainly prompt engineering.

In the second section, which is focused on the ethical and security side of the topic, our aim is to clearly explain to users how they should use AI technologies. In addition to written explanations, practical examples of suggestions will also be shown.

In the third section, we will focus on the issue of transparency in AI systems. We will suggest ways for companies and developers on how they can improve the transparency of their systems.

Using current standards and best practices in the field of artificial intelligence, these guidelines will offer a set of suggestions for the ethical and secure usage of large language models. With the proposed guidelines, our aim is to minimize risks and help increase understanding of these systems with ethics and security in mind. As mentioned before, the guidelines are aimed at developers and the general public and hopefully will be a practical and helpful tool for them.
